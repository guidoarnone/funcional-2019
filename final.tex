\documentclass[11pt]{report}
\usepackage[margin=1in]{geometry} 
\usepackage{amsmath,amsthm,amssymb,amsfonts}
\usepackage[utf8]{inputenc}
\usepackage[T1]{fontenc}
\usepackage{tocbibind}
\usepackage[spanish]{babel}
\usepackage{microtype}
\usepackage{mathpazo}
\usepackage{euler}
\usepackage{thmtools,xcolor}
\usepackage{tikz}
\usepackage{tikz-cd}
\usetikzlibrary{arrows}
\usetikzlibrary{matrix}
\usepackage{fancyhdr}
\pagestyle{fancy}
\usepackage{enumitem}
\usepackage{tcolorbox}
\tcbuselibrary{theorems}

\addto\captionsspanish{\renewcommand{\chaptername}{Parte}}
\renewcommand\qedsymbol{$\paint{\blacklozenge}$}
\definecolor{color}{RGB}{75, 150, 80}
\declaretheoremstyle[
  headfont=\color{color}\normalfont\bfseries,
  notefont=\color{color}\normalfont\bfseries
]{colored}
\theoremstyle{colored}
\newtheorem{definition}{Definición}[section]
\newtheorem{theorem}{Teorema}[section]
\newtheorem*{theorem*}{Teorema}
\newtheorem{proposition}{Proposición}[section]
\newtheorem{corollary}{Corolario}[section]
\newtheorem{lemma}{Lema}[section]
\newtheorem{remark}{Observación}[section]

\newcommand{\N}{\mathbb{N}}
\newcommand{\Z}{\mathbb{Z}}
\newcommand{\Q}{\mathbb{Q}}
\newcommand{\R}{\mathbb{R}}
\newcommand{\C}{\mathbb{C}}
\newcommand{\M}[2]{\mathsf{M}_{#1}#2}
\newcommand{\im}{\operatorname{im}}
\newcommand{\sop}{\operatorname{sop}}
\newcommand{\ev}{\operatorname{ev}}
\newcommand{\id}{\operatorname{id}}
\newcommand{\eps}{\varepsilon}
\newcommand{\nat}[1]{[\![#1]\!]}
\newcommand{\natzero}[1]{\nat{#1}_0}
\newcommand{\ip}[1]{( #1 )}
\newcommand{\ol}{\overline}
\newcommand{\paint}[1]{\color{color}{#1}}
\newcommand{\tpaint}[1]{\paint{\textbf{#1}}}
\newcommand{\paintline}{\begin{center}
$\paint{
\rule{400pt}{0.5pt}
}$
\vspace{10pt}
\end{center}}

%-----------------------

\title{
\LARGE{Análisis Funcional}
\\
\vspace{1pt}
\small{Primer Cuatrimestre -- 2019}
\\
\vspace{0.5pt}
\large{Examen Final}
\\
\vspace{80pt}
{\includegraphics[height=5cm]{uba2.jpg}}
\vspace{80pt}
}
\author{Guido Arnone}
\date{}
\lhead{Guido Arnone}
\rhead{Examen Final}

\begin{document}

\maketitle
\tableofcontents

\chapter{Preliminares}
Recuerdo primero algunos resultados que vimos en la materia y serán necesarios para la demostración del teorema espectral.
\subsection{Proyectores, Teoremas de Representación y Sumas Hilbertianas}

\begin{theorem}[de la proyección ortogonal] Sea $H$ un espacio de Hilbert y $K \subset H$ un subcojunto convexo, cerrado y no vacío. Dado $f \in H$, existe un único $u \in K$ tal que
\begin{align*}
\|f-u\| = \min_{v \in K}\|f-v\|.
\end{align*}
Más aún, el vector $u$ se caracteriza por satisfacer
\begin{align*}
\begin{cases}
u \in K\\
\ip{f-u,v-u} \leq 0 \quad (\forall v \in K)
\end{cases}
\end{align*}
Notamos $P_Kf := u$.
\end{theorem}
\begin{proof} Sea $\{v_n\}_{n \geq 1} \subset K$ una sucesión que realiza $d := \inf_{v \in K}\|f-v\|$ de forma decreciente y veamos que $\{v_n\}_{n \geq 1}$ es de Cauchy. Notemos $d_n := \|f-v_n\|$ para cada $n \in \N$. Usando la identidad del paralelogramo, es
\begin{align}
\left\|f-\frac{v_n+v_m}{2}\right\|^2 + \left\|\frac{v_n-v_m}{2}\right\|^2 = \frac{d_n^2 + d_m^2}{2}
\end{align}
para cada $n,m \in \N$. Como $K$ es convexo, sabemos que $1/2(v_n+v_m) \in K$ y por lo tanto  es $\|f-\frac{v_n+v_m}{2}\|^2 \geq d^2$. De $\tpaint{(1.1)}$ tenemos así 
\begin{align*}
\|v_n-v_m\|^2 \leq 4\left(\frac{d_n^2+d_m^2}{2}-d^2\right),
\end{align*}
lo que efectivamente muestra que $\{v_n\}_{n \geq 1}$ es de Cauchy. Existe entonces un cierto límite $u \in K$ que realiza $d(f,K)$. 

Sea ahora $v \in K$ y $v_t = tv + (1-t)u$. Como $\varphi(t) = \|f-v_t\|^2$ se minimiza en $0$, es $0  \leq \varphi'(0) = -2(f-u,v-u)$ y entonces $(f-u,v-u) \leq 0$. Recíprocamente, si $u \in K$ es tal que $(f-u, v-u) \leq 0$ para todo $v \in K$ entonces 
\begin{align}
\|f-u\|^2 -\|f-v\|^2 = 2(f-u,v-u) \leq 0
\end{align}
lo que dice que $u$ realiza $d(f,K)$.

De esta caracterización vemos la unicidad: si $u$ y $u'$ minimizan $d(f,K)$ es entonces
\begin{align*}
\|u-u'\|^2 = \ip{u-f,u-u'} + \ip{f-u',u-u'} = \ip{f-u,u'-u} + \ip{f-u',u-u'} \leq 0,
\end{align*}
lo que muestra que $u = u'$.
\end{proof}

\begin{corollary} Sea $H$ un espacio de Hilbert y $K \subset H$ un convexo cerrado no vacío. Entonces $P_K$ es continua y $\|P_Kf_1 - P_Kf_2\| \leq \|f_1 - f_2\|$.
\end{corollary}
\begin{proof} Sea $u_i = P_Mf_i$. Como es $(f_i - u_i,v-u_i) \leq 0$ para todo $v \in K$, tenemos que
\[
(f_2-f_1,u_1-u_2) + \|u_1-u_2\|^2 \leq 0.
\]
Ahora, por la desigualdad de Cauchy-Schwartz es
\[
\|u_1-u_2\|^2 \leq \ip{f_1-f_2,u_1-u_2} \leq \|f_1-f_2\|\|u_1-u_2\|,
\]
de forma que $\|P_Kf_1-P_Kf_2\| = \|u_1-u_2\| \leq \|f_1-f_2\|$.
\end{proof}

\begin{corollary} Sea $H$ un espacio de Hilbert y $M \leq H$ un subespacio cerrado. La aplicación $f \in H \mapsto P_Mf \in H$ es un operador continuo. Más aún, $P_M$ es un proyector y para cada $f \in H$ el vector $P_Mf$ se caracteriza como el único tal que $\ip{f-u,v} = 0$ para todo $v \in M$.
\end{corollary}
\begin{proof} La linealidad es consecuencia de la caracterización de $P_Mf$ que probamos a continuación, mismo el hecho de que $P_M$ es un proyector. La suficiencia está dada por el $\tpaint{Teorema 1.0.1}$, así que veremos la necesidad. 

Como para cada $v \in M$ es $P_Mf \pm v \in M$, tenemos que
\begin{align*}
0 \geq (f-P_Mf, (P_Mf \pm v) - P_Mf) = \pm(f-P_Mf,v)
\end{align*}
y por lo tanto $(f-P_Mf,v) = 0$. 
\end{proof}

\begin{theorem}[de representación de Riesz] Sea $H$ un espacio de Hilbert. Sin $\varphi \in H^*$ es un funcional lineal, existe un único $u \in H$ tal que
\begin{align*}
\langle \varphi, v\rangle  = \ip{u,v}
\end{align*}
para todo $v \in H$.
\end{theorem}
\begin{proof} Basta ver que $j : v \in H \mapsto \ip{v,-} \in H^*$ tiene imagen densa. Sea $\Phi : H^* \to \R$ un funcional que se anula en $j(H)$. Como $H$ es reflexivo, sabemos que $\Phi = \ev_{x}$ para cierto $x \in H$ y por lo tanto es
\begin{align*}
0 = \Phi(j(y)) = j(y)(x) = \ip{y,x}
\end{align*}
para todo $y \in H$. En consecuencia debe ser $x = 0$ y $\Phi \equiv 0$.
\end{proof}

\begin{proposition} Sea $H$ un espacio de hilbert y $S \leq H$ un subespacio. 
\begin{itemize}
\item[(i)] $S^\perp$ es cerrado.
\item[(i)] $S^{\perp\perp} = \overline{S}$.
\item[(iii)] $S^\perp = H$ si y sólo si $S = \{0\}$.
\item[(iv)] $S$ es denso si y sólo si $S^\perp = \{0\}$
\end{itemize}
\end{proposition}
\begin{proof} Hacemos cada inciso por separado.
\begin{itemize}[listparindent = \parindent]
\item[(i)] Sea $\{x_n\}_{n \geq 1} \subset S^\perp$ tal que $x_n \to x$. Como para cada $y \in S$ es 
\[
\ip{x,y} = \ip{\lim_{n \to \infty}x_n,y} = \lim_{n \to \infty}\ip{x_n,y} = 0,
\]
debe ser $x \in S^\perp$. Por lo tanto, el ortogonal de $S$ es cerrado.
\item[(ii)] Basta ver que $S$ es denso en $S^{\perp\perp}$. Sea $\varphi : S^{\perp \perp} \to \R$ es un funcional que se anula en $S$. Por el teorema de representación de Riesz, es $\varphi \equiv \ip{u,-}$ para cierto $u \in H$. Como además $\varphi$ se anula en $S$, sabemos que $u \in S^\perp$ y por lo tanto $\varphi \equiv 0$. 
\item[(iii)] En efecto, $S^\perp = H$ si y sólo si para todo $h \in H$ y $s \in S$ es
\[ (s,h) = 0, \]
lo que equivale a decir $s = 0$ para todo $s \in S$.
\item[(iv)] Resta notar que $S$ es denso si y sólo si $S^{\perp \perp} = \overline{S} = H$, y por $\tpaint{(iii)}$ esto ocurre si y sólo si $S^\perp = \{0\}$.
\end{itemize}
\end{proof}

\begin{definition} Sea $H$ un espacio de Hilbert y $\mathfrak{a} : H \times H \to \R$ una función bilineal. Decimos que $\mathfrak{a}$ es
\begin{itemize}
\item[$\paint{\bullet}$] \textbf{continua} si existe $C \geq 0$ tal que $|\mathfrak{a}(x,y)| \leq C \|x\|\|y\|$ para todo $x,y \in H$. 
\item[$\paint{\bullet}$] \textbf{cohesiva} si existe $\theta > 0$ tal que $\mathfrak{a}(x,x) \geq \theta \|x\|^2$ para todo $x  \in H$.
\end{itemize}
\end{definition}
\begin{remark} Si $\mathfrak{a}$ es una función bilineal continua y cohesiva en un espacio de Hilbert, induce un producto interno equivalente al original.
\end{remark}

\begin{theorem}[Stampacchia] Sea $H$ un espacio de Hilbert y $\mathfrak{a} : H \times H \to \R$ una función bilineal continua y cohesiva. Si $K \subset H$ es convexo cerrado y no vacío y $\varphi \in H^*$ un funcional lineal, entonces existe un único vector $u \in K$ tal que
\begin{align*}
\mathfrak{a}(u,v-u) \geq \langle \varphi, v-u \rangle \quad (\forall v \in K)
\end{align*}
Si además $\mathfrak{a}$ es simétrica, el vector $u$ se caracteriza por
\begin{align*}
\begin{cases}
u \in K\\
\frac{1}{2}\mathfrak{a}(u,u) - \langle \varphi, u \rangle = \inf_{v \in K}\frac{1}{2}\mathfrak{a}(v,v) - \langle \varphi, v \rangle
\end{cases}
\end{align*}
\end{theorem}
\begin{proof} Por el lema de Riesz existe un único $f \in H$ tal que $\varphi \equiv \ip{f,-}$ y, para cada $u \in H$, existe un único $Au \in H$ tal que $\mathfrak{a}(u,-) \equiv \ip{Au,-}$. 

Por unicidad, la aplicación $A : H \to H$ resulta lineal. Más aún es continua pues
\begin{align*}
\|Au\|^2 = \ip{Au,Au} = \mathfrak{a}(u,Au) \leq C \|u\| \cdot \|Au\|, y
\end{align*}
como $\mathfrak{a}$ es cohesiva tenemos que $\ip{Au,u} \geq \theta \|u\|^2$.

Reescribiendo, lo que debemos ver es que existe un único $u \in K$ tal que
\begin{align*}
(Au, v-u) \geq (f,v-u)
\end{align*}
para todo $v \in K$, o equivalentemente que para cierto $\rho > 0$ es
\begin{align*}
\ip{\rho(f-Au) + u -u, v-u} \leq 0 \quad (\forall v \in K).
\end{align*}

Alcanza entonces ver que para algún $\rho > 0$ $S(u) := P_K(\rho(f-Au)+u)$ es estrictamente contractiva. En efecto, como es
\begin{align*}
\|Su-Sv\|^2 &\leq \|(u-v) - \rho A(u-v)\| = \|u-v\|^2 -2\rho(u-v,A(u-v)) + \rho^2\|A(u-v)\|^2\\
&\leq \|u-v\|^2 - 2\rho \theta \|u-v\|^2 + \rho^2C^2\|u-v\|^2\\
&= (C^2\rho^2-2\theta\rho+1)\|u-v\|^2,
\end{align*}
basta tomar $\rho > 0$ tal que 
\[
C^2\rho^2-2\theta\rho+1 < 1 \iff C^2\rho^2-2\theta\rho < 0 \iff \rho < \frac{2\theta}{C^2}.
\]
\end{proof}

\begin{theorem}[Lax-Milgram] Sea $H$ y $\mathfrak{a} : H \times H \to \R$ una función bilineal continua, cohesiva y simétrica. Si $\varphi \in H^*$ es un funcional lineal, entonces existe un único $u \in H$ tal que $\mathfrak{a}(u,-) \equiv \varphi$.
\end{theorem}
\begin{proof} Por el teorema de Stampacchia, sabemos que existe $u \in H^*$ tal que
\[
\mathfrak{a}(u,v) - \langle \varphi, v \rangle \geq 0
\]
para todo $v \in H$. Como $\mathfrak{a}(u,-) - \varphi$ en un funcional lineal acotado en $H$, debe ser cero, y por lo tanto tenemos que $\mathfrak{a}(u,-) = \varphi$.
\end{proof}

\begin{definition} Sea $H$ un espacio de Hilbert y $(E_n)_{n \geq 1}$ una sucesión de subespacios cerrados de $H$. Se dice que $H$ es \textbf{suma hilbertiana} de $(E_n)_{n \geq 1}$ si 
\begin{itemize}
\item[$\paint{\bullet}$] $E_i \perp E_j$ si $i \neq j$, y
\item[$\paint{\bullet}$] $\operatorname{gen} \ \{E_n\}_{n \geq 1}$ es denso.
\end{itemize}
Notamos $H = \bigoplus_{n = 1}^\infty E_n$.
\end{definition}

\begin{theorem} Sea $H$ un espacio de Hilbert con $H = \bigoplus_{n = 1}^\infty E_n$ y $u \in H$. Si notamos $u_n = P_{E_n}u$ para cada $n \in \N$, entonces
\begin{itemize}
\item[(i)] $u = \sum_{n \geq 1}u_n$.
\item[(ii)] $\|u\|^2 = \sum_{n \geq 1}\|u_n\|^2$.
\end{itemize}

Recíprocamente, si tomamos $u_n \in E_n$ para cada $n \in \N$ y es $\sum_{n \geq 1}\|u_n\|^2 < \infty$, entonces $u := \sum_{n \geq 1}u_n$ converge y se tiene que $u_n = P_{E_n}u$ para cada $n \in \N$.
\end{theorem}
\begin{proof} Para cada $k \in \N$, definimos $S_k := \sum_{n=1}^k P_{E_n} \in \mathscr{L}(H)$. Por ortogonalidad se tiene que
\begin{align*}
\|S_ku\|^2 = \sum_{n=1}^k\|u_n\|^2 = \sum_{n=1}^k(u,u_n) = (u,S_ku).
\end{align*}
pues definición de proyector $P_{E_n}$, es $(u_n,u_n-u) = 0$ y por tanto $\|u_n\|^2 = \ip{u,u_n}$. Usando la desigualdad de Cauchy-Schwartz obtenemos
\begin{align*}
\|S_ku\|^2 \leq \|u\|\|S_ku\|,
\end{align*}
por lo que debe ser $\|S_k\| \leq 1$.

Ahora, fijemos $\eps > 0$. Por densidad existe $u_\eps \in \operatorname{gen}\ \{E_n\}_{n \geq 1}$ tal que $\|u-u_\eps\|< \eps$ y $k_0 \in \N$ tal que $S_ku_\eps = u_\eps$ si $k > k_0$. Por lo tanto, para todo $k > k_0$ es
\begin{align*}
\|S_ku-u_\eps\| = \|S_k(u-u_\eps)\| \leq \|u-u_\eps\| < \eps 
\end{align*}
y
\begin{align*}
\|S_ku-u\| \leq \|S_ku-u_\eps\| + \|u_\eps - u\| < 2\eps.
\end{align*}

En otras palabras, vemos que $\sum_{n \geq 1}u_n = \lim_{n \to \infty} S_ku = u$. De aquí es también que
\begin{align*}
\|u\|^2 = \|\lim_{n \to \infty}S_ku\|^2 = \lim_{n \to \infty}\|S_ku\|^2 = \sum_{n \geq 1}\|u_n\|^2.
\end{align*}

Para terminar veamos el recíproco. Si tomamos $u_n \in E_n$ para cada $n \in \N$ tales que $\sum_{n \geq 1}\|u_n\|^2 < \infty$, entonces notando $s_k = \sum_{n=1}^ku_n$ por ortogonalidad vemos que
\begin{align*}
\|s_k-l\|^2 \leq \sum_{l < n \leq k}\|u_n\|^2.
\end{align*}
Esto dice que $(s_k)_{k \geq 1}$ es de Cauchy, y en consecuencia $\lim_{k \to \infty}s_k = \sum_{n \geq 1}u_n$ existe. Por último, resta notar que por la continuidad de los proyectores es
\begin{align*}
P_{E_m}(u) = \sum_{n \geq 1}P_{E_m}(u_n) = u_m
\end{align*} 
para todo $m \in \N$.
\end{proof}

\begin{definition} Sea $H$ un espacio de Hilbert. Una sucesión $\{e_n\}_{n \geq 1}$ se dice una $\textbf{base hilbertiana}$ si
\begin{itemize}
\item[$\paint{\bullet}$] $\ip{e_n,e_m} = \delta_{nm}$ para todo $n,m \in \N$, y
\item[$\paint{\bullet}$] $\operatorname{gen} \ \{e_n\}_{n \geq 1}$ es denso.
\end{itemize}
\end{definition}

\begin{corollary} Sea $H$ un espacio de Hilbert. Si $\{e_n\}_{n \geq 1} \subset H$ es una sucesión ortonormal, entonces esta es una base hilbertiana si y sólo si 
\begin{align*}
u = \sum_{n \geq 1}(u,e_n)e_n \text{ y } \|u\|^2 = \sum_{n \geq 1}|(u,e_n)|^2.
\end{align*}
para todo $u \in H$.

Recíprocamente, si $(\alpha_n)_{n \geq 1} \subset \ell^2$ entonces la serie $\sum_{n \geq 1}\alpha_n e_n$ converge en $H$ a un elemento, y su norma es exactamente $\sum_{n \geq 1}\alpha_n^2$. \qed
\end{corollary}

\begin{remark} Si $H$ admite una base Hilbertiana $\{e_n\}_{n \geq 1}$, la aplicación $u \in H \mapsto \{(u,e_n)\}_{n \geq 1} \in \ell^2$ es un isomorfismo isométrico.
\end{remark}

\begin{theorem} Un espacio de Hilbert separable de dimensión infinita admite una base hilbertiana.
\end{theorem}
\begin{proof} Sea $\{v_n\}_{n \geq 1} \subset H$ denso y $F_k = \langle v_1, \dots, v_k \rangle$ para cada $k \in \N$. Por lo tanto, es
\begin{align*}
\overline{\bigcup_{k \geq 1}F_k} = H.
\end{align*}

Para cada $k \geq 1$, podemos tomar una base ortonormal $B_{k+1}$ de $F_{k+1}$ que extienda una base ortonormal de $F_k$. Tomando $B = \bigcup_{k \geq 1}B_k$ obtenemos así una base hilbertiana de $H$.
\end{proof}

\subsection{Operadores Compactos}

\begin{definition} Sean $E$ y $F$ dos espacios de Banach. Un operador $T \in \mathscr{L}(E,F)$ se dice \textbf{compacto} si $\overline{T(B_E)}$ es compacto. Equivalentemente, el operador $T$ es compacto si para toda sucesión acotada $\{x_n\}_{n \geq 1} \subset E$ la sucesión $\{Tx_n\}_{n \geq 1} \subset F$ es precompacta.
\end{definition}

\begin{proposition} Si $E$ y $F$ dos espacios de Banach, el conjunto $\mathscr{K}(E,F)$ es un subsepacio cerrado de $\mathscr{L}(E,F)$.
\end{proposition}
\begin{proof} Supongamos que $T_{n} \rightrightarrows T$ para cierta sucesión $\{T_n\}_{n \geq 1} \subset \mathscr{K}(E)$ de operadores compactos. Veamos que $\overline{T(B_E)}$ es compacto, o equivalentemente, que $T(B_E)$ es totalmente acotada.

Fijemos $\eps > 0$ y tomemos $n_0 \in \N$ tal que si $n > n_0$ entonces $\|T-T_{n_0}\| < \eps/2$. Como $T_{n_0}$ es un operador compacto, existen $f_1, \dots, f_j \in E$ tales que
\begin{align*}
T_{n_0}(B_E) \subset \bigcup_{s=1}^{j}B_{\eps/2}(f_s).
\end{align*}
Afirmamos entonces que $T(B_E) \subset \bigcup_{s=1}^{j}B_{\eps}(f_s)$. En efecto, si $x \in B_E$, entonces existe $s \in \natzero{j}$ tal que $T_{n_0}x \in B_{\eps/2}(f_s)$ y por lo tanto, es
\begin{align*}
\|Tx-f_s\| \leq \|Tx-T_{n_0}x\| + \|T_{n_0}x - f_s\| < \eps/2 + \eps/2 = \eps.
\end{align*}
\end{proof}

\begin{corollary} Sean $E$ y $F$ son espacios de Banach. Si $T \in \mathscr{L}(E,F)$ es un operador que es límite de operadores de rango finito, entonces es compacto.
\end{corollary}
\begin{proof} Como los operadores compactos forman un subespacio cerrado, resta notar que un operador de rango finito siempre es compacto.
\end{proof}

\begin{theorem} Sean $E$ un espacio de Banach y $H$ un espacio de Hilbert. Si $T \in \mathscr{L}(E,H)$ es un operador acotado, entonces existe una sucesión $(T_n)_{n \geq 1} \subset  \mathscr{L}(E,H)$ de operadores de rango finito tal que $T_n \rightrightarrows T$.
\end{theorem}
\begin{proof} Veamos equivalentemente que los operadores de rango finito son densos en los operadores compactos.

Sea $\eps > 0$. Como $T$ es compacto, existen $f_1, \dots, f_j \in E$ tales que
$\overline{T(B_E)} \subset \bigcup_{s=1}^jB_\eps(f_j)$. Definamos ahora $G = \langle f_1, \dots, f_j \rangle$. Luego $T_\eps = P_{G_\eps}T$ es de rango finito y si $x \in B_E$ con $Tx \in B_\eps(f_s)$, entonces 
\begin{align*}
\|T_\eps x - Tx\| &\leq \|T_\eps x - f_s\| + \|f_s -Tx\| = \|P_{G_\eps}(Tx - f_s)\| + \|f_s -Tx\|\\
&\leq \|P_{G_\eps}\|\|Tx - f_s\| + \|f_s - Tx\| \leq 2\eps,
\end{align*}
lo que prueba que $\|T_\eps -T\| \leq 2\eps$.
\end{proof}

\begin{remark} Sean $E,F$ y $G$ espacios de Banach y $T \in \mathscr{L}(E,F), S \in \mathscr{L}(F,G)$ operadores acotados. Si $S$ o $T$ son compactos, $ST$ lo es.
\end{remark}

\begin{theorem}[Alternativa de Fredholm] Sea $E$ un espacio de Banach y $T \in \mathscr{K}(E)$ un operador compacto. Entonces
\begin{itemize}
\item[(a)] $\dim N(I-T) < \infty$.
\item[(b)] $R(I-T)$ es cerrado y $R(I-T) = {}^\perp N(I-T^*)$.
\item[(c)] $N(I-T) = \{0\} \iff R(I-T) = E$.
\item[(d)] $\dim N(I-T^*) = \dim N(I-T)$.
\end{itemize} 
\end{theorem}
\begin{proof} Probaremos $\tpaint{(a)$ y $(b)}$, que es lo necesario para demostrar el teorema espectral (de aquí se deduce $\tpaint{(c)}$ si $T$ es autoadjunto). La demostración completa se encuentra, por ejemplo, en $\cite{Brezis}$.
\begin{itemize}[listparindent = \parindent]
\item[(a)] Veamos equivalentemente que $B_{E_1}(0)$ es compacto. Si $x \in N(I-T)$, sabemos que $x = Tx \in T(B_E(0))$ y entonces $B_{E_1} \subset T(B_E(0))$ es un cerrado contenido en un compacto, esto es, un compacto.
\item[(b)] Ya sabemos que $\overline{R(I-T)} = {}^\perp N(I-T^*)$, basta ver que $I-T$ tiene rango cerrado. Tomemos una sucesión $(f_n)_{n \geq 1} \subset R(I-T)$, con $f_n = u_n - Tu_n$ y $u_n \in E$ para cada $n \in \N$, que converge a cierto $f \in E$.

Notemos $d_n = d(u_n, N(I-T))$ para cada $n \in \N$. Como $N(I-T)$ es de dimensión finita, existe $v_n \in N(I-T)$ tal que $d_n = \|u_n-v_n\|$. Rescribiendo, es
\begin{align*}
f_n &= u_n - Tu_n = (u_n - v_n) + v_n - Tu_n = (u_n - v_n) + Tv_n - Tu_n\\
&= (u_n-v_n) -T(u_n-v_n).
\end{align*}

Veamos ahora que $\{u_n-v_n\}_{n \geq 1}$ es acotada: de lo contrario, tendríamos una sucesión tal que $\|u_{n_k}-v_{n_k}\| \to \infty$. Notando $w_k := \frac{u_{n_k}-v_{n_k}}{\|u_{n_k}-v_{n_k}\|}$ tenemos que $w_k - Tw_k = d_{n_k}^{-1}f_{n_k}$ y $\|w_k\| = 1$ para todo $k \in \N$. 

Como $f_n \to f$ y $d_{n_k} \to \infty$, debe ser $d_{n_k}^{-1}f_{n_k} \to 0$ y al $T$ ser compacto, podemos además tomar la subsucesión de forma que $Tw_k \to z$ para cierto $z \in N(I-T)$, por lo que necesariamente es $w_k \to z$. Sin embargo esto es absurdo, pues para todo $k \in \N$ es $d(w_k,N(I-T)) = 1$, y vemos así que $\{u_n-v_n\}_{n \geq 1}$ es acotada.

Una vez más, por la compacidad $T$ existe una sucesión convergente $T(u_{n_k}-v_{n_k}) \to \ell$ y entonces es $u_{n_k}-v_{n_k} \to f+\ell$. Notando $g = f+l$, finalmente es
\[
f = g -Tg \in R(I-T)
\]
y por lo tanto $I-T$ tiene rango cerrado.
\end{itemize}
\end{proof}

\subsection{Teoría Espectral}

\begin{definition} Sea $E$ un espacio de Banach y $T \in \mathscr{L}(E)$ un operador acotado.
El \textbf{espectro} de $T$ es el conjunto
\begin{align*}
\sigma(T) := \{\lambda \in \R : T-\lambda I \text{ no es inversible} \},
\end{align*}
y el \textbf{espectro puntual} es
\begin{align*}
\sigma_p(T) = \{\lambda \in \R : \ker(T-\lambda I) \neq \{0\} \ \}.
\end{align*}
Definimos también la \textbf{resolvente} de $T$ como $\rho(T) := \R \setminus \sigma(T)$.
\end{definition}

\begin{proposition} Si $E$ un espacio de Banach y $T \in \mathscr{L}(E)$ un operador acotado, el espectro de $T$ es compacto y $\sigma(T) \subset [-\|T\|,\|T\|]$.  
\end{proposition}
\begin{proof} Veamos primer que $\sigma(T) \subset [-\|T\|,\|T\|]$. Consideremos $\lambda \in \R$ tal que $|\lambda| > \|T\| \geq 0$ y veamos que $\lambda \in \rho(T)$.

Dicho de otra forma, veamos que para todo $y \in E$ la ecuación
\begin{align*}
x = \frac{1}{\lambda}(Tx - y)
\end{align*}
tiene solución única. En efecto, por el teorema de punto fijo de Banach basta notar que la aplicación $J(x) := \frac{1}{\lambda}(Tx-y)$ es una contracción estricta, pues dados $u,v \in E$ es
\begin{align*}
\|J(u)-J(v)\| = \frac{1}{|\lambda|}\|T(u-v)\| \leq \frac{\|T\|}{|\lambda|}\|u-v\|
\end{align*}
y por hipótesis sabemos que $\frac{\|T\|}{|\lambda|} < 1$.

Para terminar, veamos que el espectro es cerrado mostrando que la resolvente es abierta. Fijemos $\lambda_0 \in \rho(T)$ y sea $\lambda \in \R$. Ahora, la aplicación
\begin{align*}
T-\lambda I = T-\lambda_0 I + (\lambda_0 - \lambda)I
\end{align*} 
será biyectiva si y sólo si para cada $y \in E$ la ecuación 
\begin{align*}
x = (T-\lambda_0 I)^{-1}y + (\lambda-\lambda_0)(T-\lambda_0 I)^{-1}x
\end{align*}
tiene solución única. Esto se satisface en particular cuando la aplicación $\widetilde{J}(x) := (T-\lambda_0 I)^{-1}y + (\lambda-\lambda_0)(T-\lambda_0 I)^{-1}x$ es contractiva, y con el mismo argumento que antes, vemos que esto se puede asegurar si
\begin{align*}
|\lambda-\lambda_0| < \|(T-\lambda_0 I)^{-1}\|^{-1}.
\end{align*}
\end{proof}

\begin{theorem} Sea $E$ un espacio de Banach de dimensión infinita. Si $T \in \mathscr{K}(E)$ es un operador compacto, entonces
\begin{itemize}
\item[(i)] $0 \in \sigma(T)$.
\item[(ii)] $\sigma(T) \setminus \{0\} = \sigma_p(T) \setminus \{0\}$.
\item[(iii)] O bien $\sigma(T) = \{0\}$, o bien $\sigma(T)$ es finito, o bien es $\sigma(T) \setminus \{0\} = \{\lambda_n\}_{n \geq 1}$ con $\lambda_n \to 0$.
\end{itemize} 
\end{theorem}
\begin{proof} Hacemos cada inciso por separado.
\begin{itemize}[listparindent = \parindent]
\item[(i)] Si $0$ no perteneciese al espectro de $T$, éste sería un operador inversible. Pero como a su vez es compacto, tendríamos que $I = T \circ T^{-1}$ es compacta, lo cual nunca ocurre en dimensión infinita.
\item[(ii)] Sea $\lambda \in \sigma(T) \setminus \{0\}$. Notemos que $T-\lambda I$ es inversible si y sólo si lo es $\lambda^{-1}T-I$. Por la Alternativa de Fredholm, éste último es biyectivo si y sólo si es inyectivo. En consecuencia $\lambda$ pertenece al espectro puntual de $T$.
\item[(iii)] Como el espectro es compacto, basta ver que para cada $n \in \N$  el conjunto
\begin{align*}
\sigma(T) \cap \left\{\lambda \in \R : |\lambda| \geq \frac{1}{n}\right\}
\end{align*}
es finito. En particular este es discreto, así que debe ser contable. De ser infinito, esto dice además que $\sigma(T)$ tiene a $0$ como punto de acumulación y podemos entonces reordenar sus elementos de forma que resulten una sucesión $\{\lambda\}_{n \in \N}$ tal que $\lambda_n \to 0$.

Si no fuera así, existiría por compacidad una sucesión $\{\lambda_n\}_{n \geq 1} \subset \sigma(T)$ tal que $\lambda_n \to \lambda \neq 0$. Veamos que esto esto es absurdo. En particular tendríamos para cada $n \in \N$ un vector unitario $e_n \in \ker (T-\lambda I)$. 

Como la colección $\{e_n\}_{n \geq 1}$ es linealmente independiente,  notando $E_n = \langle e_1, \dots, e_n \rangle$ obtenemos una sucesión estrictamente creciente de subespacios cerrados que satisfacen $(T-\lambda_nI)(E_n) \subset E_{n-1}$.

Por el lema de Riesz, existen vectores unitarios $u_{n+1} \in E_n$ tales que
\begin{align*}
\frac{1}{2} \leq d(u_{n+1},E_n)
\end{align*}
para todo $n \geq 1$, y podemos definir entonces $v_n := \lambda_n^{-1}u_n$. Notemos que como $\{\lambda_n\}_{n \geq 1}$ converge y los vectores $\{u_n\}_{n \geq 1}$, esta es una sucesión acotada. Sin embargo, dados $1 < m < n$ se tiene que
\begin{align*}
\|Tv_m-Tv_n\| = \left\|\overbrace{\lambda_n^{-1}(T-\lambda_nI)u_n}^{\in E_{n-1}} - \overbrace{\lambda_m^{-1}(T-\lambda_mI)u_m}^{\in E_{m-1} \subset E_{n-1}} + u_n - u_m \right\| \geq d(u_n,E_{n-1}) \geq \frac{1}{2},
\end{align*}
lo que contradice la compacidad de $T$.
\end{itemize}
\end{proof}

\begin{definition} Sea $H$ un espacio de Hilbert y $T \in \mathscr{L}(H)$ Decimos que $T$ es \textbf{autoadjunto} si para todo $x,y \in H$ se tiene que $\ip{Tx,y} = \ip{x,Ty}$.
\end{definition}

\begin{theorem} Sea $H$ es un espacio de Hilbert y $T \in \mathscr{L}(H)$ un operador autoadjunto. Notando
\begin{align*}
m = \inf_{\|x\| = 1}(Tx,x) \ \text{ y } \ M = \sup_{\|x\| = 1}(Tx,x),
\end{align*}
se tiene que $\sigma(T) \subset [m,M]$ y $m,M \in \sigma(M)$. Más aún, es $\|T\| = \max\{\|m\|,\|M\|\}$.
\end{theorem}
\begin{proof} Por simetría (tomando $-T$) basta probar las afirmaciones sobre $M$. En primer lugar, sea $\lambda > M$ y veamos que $\lambda \in \rho(T)$. Como para todo $x \in H$ es
\begin{align*}
\ip{Tx,x} \leq M \|x\|^2,
\end{align*}
se tiene que
\begin{align*}
(\lambda x - Tx,x) \geq (\lambda - M)\|x\|^2.
\end{align*}

Al ser $\lambda - M > 0$, el cálculo anterior nos dice que la forma bilineal
\begin{align*}
\mathfrak{a} : &H \times H \to \R\\
&(x,y) \mapsto (\lambda x -T x, y)
\end{align*}
es continua y cohesiva. El teorema de Stampacchia nos asegura entonces que para todo $y \in H$ existe un único elemento $x \in H$ tal que $\mathfrak{a}(x,-) \equiv \ip{y,-}$.

Dicho de otra forma, la ecuación
\begin{align*}
(\lambda I -T)x = y
\end{align*}
tiene solución única para todo $y \in H$, y en consecuencia $T- \lambda I$ es inversible.

Veamos ahora que $M \in \sigma(T)$. Definimos ahora $\mathfrak{a}(x,y) := (Mx-Tx,y)$. Como $T$ es autoadjunta sabemos que $\mathfrak{a}$ es simétrica y positiva. Esto dice que esta función \textit{«debe satisfacer Cauchy-Schwarz»}, 
\begin{align*}
|\mathfrak{a}(x,y)| \leq \mathfrak{a}(x,x)^{1/2} \cdot \mathfrak{a}(y,y)^{1/2} \quad (\forall x,y \in H).
\end{align*}

Como $\mathfrak{a}(y,y) \leq (|M|+\|T\|)\|y\|^2$, poniendo $y = Mx-Tx$ es
\begin{align*}
\|Mx-Tx\|^2 \leq \mathfrak{a}(x,x)^{1/2}(|M|+\|T\|)^{1/2} \cdot \|Mx-Tx\|
\end{align*}
y  notando $C = (|M|+\|T\|)^{1/2}$ en definitiva obtenemos que
\begin{align*}
\|Mx-Tx\| \leq C \cdot \mathfrak{a}(x,x)^{1/2},
\end{align*}
para todo $x \in H$.

Si ahora tomamos $\{x_n\}_{n \geq 1}$ unitarios tales que $\ip{Tx_n,x_x} \to M$, vemos que $MI-T$ no está acotado inferiormente pues
\begin{align*}
\|(M I -T )x_n\| \leq \mathfrak{a}(x_n,x_n)^{1/2} \to 0.
\end{align*}
Consecuentemente $T-M I$ no puede ser inversible, o lo que es lo mismo, debe ser $M \in \sigma(T)$.

Por último, si definimos $\mu := \max\{|m|,|M|\}$ entonces dados $x,y \in H$, al expandir $\mathfrak{a}(v,v)$ para $v \in \{x+y,x-y\}$ y restar vemos que
\begin{align*}
4\ip{Tx,y}  = \ip{T(x+y),x+y}-\ip{T(x-y),x-y} \leq M\|x+y\|^2 - m\|x-y\|^2
\end{align*}
para todo $x,y \in H$. Desarrollando el lado derecho, llegamos a 
\begin{align*}
|\ip{Tx,y}| \leq \mu \left(\frac{\|x\|^2 + \|y\|^2}{2}\right)
\end{align*}
y más aún, si $0 \neq \alpha \in \R$ debe ser
\begin{align*}
|\ip{Tx,y}| = |\ip{T\alpha x, \alpha^{-1}y}| = \mu \left(\alpha^2\frac{\|x\|^2 + \alpha^{-2}\|y\|^2}{2}\right).
\end{align*}

Si $\|x\| \neq 0$, tomando $\alpha = \|y\|/\|x\|$ es $|\ip{Tx,y}| \leq \mu \|x\|\|y\|$ y finalmente se obtiene
\begin{align*}
\|T\| = \sup_{\|x\| = 1}\|Tx\| = \sup_{\|x\|,\|y\| = 1} |(Tx,y)| \leq \mu.
\end{align*}
\end{proof}

\begin{corollary} Sea $H$ un espacio de Hilbert y $T \in \mathscr{L}(H)$ un operador autoadjunto. Si $\sigma(T) = \{0\}$, es $T = 0$. \qed
\end{corollary}

\chapter{El teorema espectral, cálculo funcional continuo y aplicaciones}
\subsection{El teorema espectral}
\begin{theorem}[espectral para operadores compactos y autoadjuntos] Sea $H$ un espacio de Hilbert separable. Si $T \in \mathscr{L}(H)$ es un operador compacto y autoadjunto, entonces existe una base ortonormal de autovectores $\{e_n\}_{n \geq 1}$ de $T$.
\end{theorem}
\begin{proof} Como $T$ es compacto, sabemos que $\sigma(T) \setminus \{0\} = \sigma_p(T) \setminus \{0\}$ y $\sigma_p(T) = \{\lambda_n\}_{n \in F}$ para cierto $F$ finito o numerable. Podemos suponer además que $\sigma(T) \neq \{0\}$, pues $T$ es nula en caso contrario. Notando $\lambda_0 := 0$ y $F_0 = F \cup \{0\}$, definimos
\begin{align*}
E_n := \ker (T - \lambda_nI)
\end{align*} 
para cada $n \in F_0$. 

Afirmamos que $H$ es la suma hilbertiana de $(E_n)_{n \in F_0}$. En primer lugar, sabemos que $E_i \perp E_j$ si $i \neq j$ pues para cada $x \in E_i$ e $y \in E_j$ es
\begin{align*}
\lambda_i(x,y) = (\lambda_ix,y) = (Tx,y) = (x,Ty) = (x,\lambda_jy) = \lambda_j(x,y),
\end{align*}
y esto implica que $\ip{x,y} = 0$. 

Por último, para concluir que $D = \operatorname{gen} \ \{E_n\}_{m \in F_0}$ es denso veamos que $D^\perp = 0$. Dado que $T(D) \subset D$, sabemos que $T(D^\perp) \subset D^\perp$ y podemos considerar entonces el operador $T_0 \equiv T|_{D^\perp}^{D^\perp}$, que es autoadjunto (y compacto). Se tiene además que $\sigma(T) = \{0\}$, ya que si $u \in D^\perp$ es tal que $T_0u = Tu = \lambda u$ para cierto $\lambda \neq 0$, es entonces $u \in D \cap D^\perp = \{0\}$.

Como $T_0$ es autoadjunto y sólo tiene a cero en su espectro, es el operador nulo. Por lo tanto, tenemos que
\begin{align*}
D^\perp \subset \ker T \subset D,
\end{align*}
lo que muestra que $D^\perp = \{0\}$.

Para terminar, notemos que para cada $n \in F$ el subespacio $E_n$ es de dimensión finita y por lo tanto posee una base ortonormal. Por otro lado, como $H$ es separable, sabemos que existe una base ortonormal $E_0 = \ker T$. En consecuencia, la unión de las bases de cada $E_n$ con $n \in F_0$ nos provee de una base de autovectores de $T$.
\end{proof}

\subsection{Cálculo Funcional}

Extendiendo la noción de \textit{«polinomios evaluados en una matriz»}, el teorema espectral nos permitira darle sentido a la expresión $f(T)$ para un operador compacto y autoadjunto $T$ y cierta clase de funciones $f$. Concretamente, 

\begin{definition} Sea $H$ un espacio de Hilbert (no necesariamente separable) y $T \in \mathscr{K}(H)$ un operador compacto y autoadjunto. Por el teorema espectral, sabemos entonces que $\sigma(T) = \{\lambda_n\}_{n \in F}$ con $F \subset \N$ contable y que $H$ es la suma hilbertiana de los autoespacios $E_n = \ker(T-\lambda_n I)$ de $T$.

Por lo tanto si $x \in H$, entonces $x = \sum_{n \geq 1}P_{E_n}(x)$ y más aún se tiene que $Tx = \sum_{n \geq 1}\lambda_n P_{E_n}(x)$ pues cada vector $P_{E_n}(x)$ es un autovector de autovalor $\lambda_n$.

En vista de lo anterior, dada $f : \sigma(T) \to \R$ una función acotada definimos \textbf{la evaluación de $f$ en $T$} como
\begin{align*}
\ev_T(f)(x) := \sum_{n \geq 1}f(\lambda_n)P_{E_n}(x).
\end{align*}
Observemos que esta función está bien definida pues $\sum_{n \geq 1}\|f(\lambda_n)P_{E_n}(x)\|^2 \leq \|f\|_\infty^2 \|x\|^2$, y más aún este argumento dice que $\|ev_T(f)\| \leq \|f\|_\infty$. 

Notar además que esta definición no depende del orden de los autovalores: cualquier reordenamiento dá el mismo operador restringiendo a cada autoespacio, y como $H$ es la suma hilbertiana de los mismos, la \textit{«nueva»} definición debe coincidir con $\ev_T(f)$.
\end{definition}

\paintline
A partir de ahora $H$ denotará un espacio de Hilbert. Fijamos también un operador $T$ compacto y autoadjunto y un orden de sus de autovectores $\sigma(T) = \{\lambda_n\}_{n \in F}$ de $T$ en el sentido anterior.
\paintline

\begin{theorem} La aplicación
\begin{align*}
\ev_T : B(\sigma(T),\R&) \to \mathscr{L}(H)\\
&f \mapsto \ev_T(f)
\end{align*}
es un morfismo de álgebras de Banach continuo que satisface $\|ev_T\| \leq 1$. Más aún, se tiene que $\ev_T(\mathsf{1}) = I$ y $\ev_T(\id) = A$. Notaremos $f(T) := \ev_T(f)$.
\end{theorem}
\begin{proof} Al definir $\ev_T(f)$ vimos que se satisface $\|\ev_T(f)\| \leq \|f\|_\infty$. Por otro lado, es
\begin{align*}
\ev_T(\mathsf{1})x = \sum_{n \geq 1}\mathsf{1}(\lambda_n)P_{E_n}(x) = \sum_{n \geq 1}P_{E_n}(x) = x
\end{align*}
y 
\begin{align*}
\ev_T(\id)x = \sum_{n \geq 1}\id(\lambda_n)P_{E_n}(x) = \sum_{n \geq 1}\lambda_nP_{E_n}(x) = \sum_{n \geq 1}TP_{E_n}(x) = T\left[\sum_{n \geq 1}P_{E_n}(x)\right] = Tx,
\end{align*}
así que $\ev_T(\mathsf{1}) = I$ y $\ev_T(\id) = T$.

La linealidad es consecuencia de la linealidad de las series: si $f,g : \sigma(T) \to \R$ son acotadas y $\mu \in \R$, entonces
\begin{align*}
(f+\mu g)(T)x &= \sum_{n \geq 1}(f+\mu g)(\lambda_n)P_{E_n}(x) = \sum_{n \geq 1}f(\lambda_n)P_{E_n}(x) + \mu\sum_{n \geq 1}g(\lambda_n)P_{E_n}(x)\\
&= f(T)x + \mu g(T)x = (f(T)+\mu g(T))x.
\end{align*}

Por último, veamos que $\ev_T$ es un morfismo de álgebras de Banach: si $f,g : \sigma(T) \to \R$, entonces
\begin{align*}
f(T)g(T)x &= f(T)\left[\sum_{n \geq 1}g(\lambda_n)P_{E_n}(x)\right] = \sum_{n \geq 1}g(\lambda_n)[f(T)P_{E_n}(x)]= \sum_{n \geq 1}f(\lambda_n)g(\lambda_n)P_{E_n}(x) = fg(T)x
\end{align*}
para todo $x \in H$.
\end{proof}

\begin{proposition} Si $f : \sigma(T) \to \R$ es una función acotada, entonces
\begin{itemize}
\item[(i)] $\sigma(f(T)) = f(\sigma(T))$.
\item[(ii)] $f(T)$ es autoadjunta.
\item[(iii)] $\|f(T)\| = \|f|_{\sigma(T)}\|_{\infty}$.
\item[(iv)] Si $f \geq 0$ entonces $f(T) \geq 0$.
\end{itemize}
\end{proposition}
\begin{proof} Hacemos cada inciso por separado. 
\begin{itemize}[listparindent = \parindent]
\item[(i)] Si $\lambda_j \in \sigma(T)$, existe $u_j \in E_j \setminus \{0\}$ y
\begin{align*}
f(T)u_j = f(\lambda_j)p_{E_j}(u_j) = f(\lambda_j)u_j,
\end{align*}
así que $f(\sigma(T)) \subset \sigma(f(T))$. 

Recíprocamente, tomemos $\lambda \not \in f(\sigma(T))$. Como esto dice que función $g(t) = (f(t)-\lambda)^{-1}$ está bien definida en $\sigma(T)$ y es allí acotada, está bien definida su evaluación $g(T)$ en $T$. Como es $g(f-\lambda) = (f-\lambda)g = \mathsf{1}$, aplicando $\ev_T$ obtenemos que
\begin{align*}
g(T)(f(T)- \lambda I) = (f(T) - \lambda I)g(T) = I.
\end{align*}
y en consecuencia $\lambda$ no pertenece al espectro de $f(T)$.
\item[(ii)] Por un cálculo directo, tomando $x,y \in H$ se tiene que 
\begin{align*}
\ip{f(T)x, y} = \sum_{n \geq 1} f(\lambda_n)\ip{P_{E_n}(x),P_{E_n}(y)}  = \ip{f(T)y,x} = \ip{x, f(T)y}.
\end{align*}
\item[(iii)] Como es $\|ev_T\| \leq 1$, ya sabemos que $\|f(T)\| \leq \|f_{\sigma(T)}\|_\infty$. En vista de $\tpaint{(i)}$ tenemos la otra desigualdad, pues acotando inferiormente por los autovectores de norma $1$ se tiene que 
\begin{align*}
\|f(T)\| = \sup_{\|x\| = 1}\|f(T)(x)\| \geq \sup_{\lambda \in \sigma(f(T))}|\lambda| = \sup_{\lambda \in f(\sigma(T))}|\lambda| = \|f|_{\sigma(T)}\|_\infty.
\end{align*}
\item[(iv)] Supongamos ahora que $f \geq 0$ y sea $x \in H$. Por definición de $f(T)$ es 
\begin{align*}
(f(T)x,x) = \sum_{n \geq 1} f(\lambda_n)\|P_{E_n}(x)\|^2 \geq 0,
\end{align*}
pues por hipótesis sabemos que $f(\lambda_n) \geq 0$ para todo $n \geq 1$.
\end{itemize}
\end{proof}

\begin{remark} Dado que el espectro es compacto, esto nos permite definir $f(T)$ para cualquier $f : X \subset \R \to \R$ continua (pero no necesariamente acotada) en $X \supset \sigma(T)$. Concretamente,  podemos precomponer a $\ev_T$ con la restricción $(-)|_{\sigma(T)} : C(X, \R) \to B(\sigma(T), \R)$. La siguiente proposición indica que en algún sentido esta aplicación (que notamos $\ev_T$ de igual manera) resulta \textit{«canónica»}.
\end{remark}

\begin{proposition} La aplicación $ev_T : \mathcal{C}(\sigma(T),\R) \to \mathscr{L}(H)$ es el único morfismo de álgebras de Banach continuo que tiene a $I$ por imagen de $\mathsf{1}$ y $T$ por imagen de $\id$.
\end{proposition}
\begin{proof} Sea $e : \mathcal{C}(\sigma(T),\R) \to \mathscr{L}(H)$ un morfismo de álgebras de Banach continuo que satisface $e(\mathsf{1}) = I$ y $e(\id) = T$. Si $p (x)= a_nx^n + \dots + a_1x+ a_0$ es la evaluación de un polinomio en $\sigma(T)$, debe ser
\begin{align*}
e(p) = \sum_{i=1}^na_ne(id)^n + a_0e(\mathsf{1}) = \sum_{i=1}^na_n\ev_T(id)^n + a_0\ev_T(\mathsf{1}) = \ev_T(p).
\end{align*}

Como el espectro es compacto, los polinomios son densos en $C(\sigma(T),\R)$. Al ser tanto $\ev_T$ como $e$ funciones que coinciden en un denso de su dominio, vemos que $e = \ev_T$.
\end{proof}

\begin{proposition} Si $f : \R \to \R$ es una función continua, entonces existe un operador compacto $S \in \mathscr{K}(H)$ tal que
\begin{align*}
f(T) = S + f(0)I.
\end{align*}
\end{proposition}
\begin{proof} Por el teorema de Stone-Weierstraß, sabemos que existe una sucesión de polinomios $(p_n)_{n \geq 1}$ tal que $p_n \to f$ uniformemente y en particular, es $p_n(0) \to f(0)$. Ahora, para cada $n \in \N$ definimos
\begin{align*}
q_n = p_n - p_n(0),
\end{align*} 
y en vista de la observación anterior, se tiene que $q_n \to f - f(0)$. Aplicando $ev_T$ y usando que ésta es continua, es
\begin{align}
q_n(T) \to (f-f(0))(T) = f(T) - f(0)I.
\end{align}

Fijemos ahora $n \in \N$. Como $q_n(0) = p_n(0) - p_n(0) = 0$, existe $r \in \R[X]$ tal que $q_n = Xr$. Por lo tanto, obtenemos $q_n(T) = (Xr)(T) = ev_T(X) \circ ev_T(r) = T \circ r(T)$. Al ser $A$ un operador compacto, el operador $q_n(T)$ es compacto para cada $n \in \N$. En vista de $\tpaint{(2.1)}$, obtenemos finalmente que el operador $f(T) - f(0)I$ es compacto. Resta notar entonces que
\begin{align*}
f(T) = (f(T)-f(0)I) + f(0)I.
\end{align*}  
\end{proof}

\begin{corollary} Si $f : \R \to \R$ es una función continua que se anula en $0$, el operador $f(T)$ resulta compacto.\qed
\end{corollary}

\begin{remark} Aún cuando $f(T)$ no es compacto, la $\tpaint{Proposición 2.0.2}$ nos dá información a través de la Alternativa de Fredholm. Por ejemplo, sabemos que en ese caso el núcleo de la evaluación es de dimensión finita.
\end{remark}

\begin{proposition} Para todo par $g : \sigma(T) \to \R$, $f : g(\sigma(T)) \to \R$ de funciones continuas se tiene que \[(f \circ g)(T) = f(g(T)).\]
\end{proposition}
\begin{proof} Fijemos una tal función $g$. Observemos en primer lugar que como es $\sigma(g(T)) = g(\sigma(T))$, la aplicación
\begin{align*}
g^* : C(\sigma(g(T)), \ &\R) \to C(\sigma(T),\R)\\
&h \longmapsto h \circ g
\end{align*}
está bien definida. Más aún, se tiene que $g^*$ es un morfismo de álgebras de Banach con $\|g^*\| \leq 1$ que satisface $g^*(\mathsf{1}) = \mathsf{1}$ y $g^*(\id) = \id$.

Consecuentemente, el morfismo $e := \ev_T \circ \ g^*$ de álgebras de Banach resulta continuo y satisface tanto $e(\mathsf{1}) = I$ como $e(\id) = g(T)$, por lo que necesariamente es $e \equiv ev_{g(T)}$. 

Finalente, evaluando en $f$ obtenemos que
\[
(f \circ g)(T) = e(f) = \ev_{g(T)}(f) = f(g(T)).
\]
\end{proof}

\subsection{Aplicaciones}

En primer lugar, veamos que todo operador $T$ compacto y autoadjunto \text{«tiene una raíz enésima»}. Esto es, para cada $n \in \N$ existe un operador $S$ tal que $S^n = T$.

\begin{theorem}  Sea $H$ un espacio de Hilbert y $T \in \mathscr{L}(H)$ un operador compacto y autoadjunto. Dado $n \in \N$, se tiene que 
\begin{itemize}
\item[(i)] Si $n$ es impar, existe un único operador $S \in \mathscr{L}(H)$ tal que $S^n = T$.
\item[(ii)] Si $n$ es par, existe un operador positivo $S \in \mathscr{L}(H)$ tal que $S^n = T$ sí y solo si $T \geq 0$. En tal caso existe un único operador $S$ positivo con esta propiedad.
\end{itemize}
Notaremos $A^{1/n} := S$ en ambos casos a este operador, que de existir resulta siempre compacto.
\end{theorem}
\begin{proof} Para cada $n \geq 1$, definimos (cuando sea posible) $f :t \in \sigma(T) \mapsto t^{1/n} \in \R$.
\begin{itemize}[listparindent = \parindent] 
\item[(i)] Sea $S = f_n(A)$. Por definición, es $S^n = f_n(T) \circ \cdots \circ f_n(T) = f_n^n(T) = id(T) = T$. Además, si $\widetilde{S}$ es tal que $\widetilde{S}^n = T$, entonces
\begin{align*}
\widetilde{S} = id(\widetilde{S}) = (f_n^n \circ (t \mapsto t^n))(\widetilde{S}) = f_n^n(\widetilde{S}^n) = f_n^n(T) = S.
\end{align*}
\item[(ii)] Recordemos que como $T$ es autoadjunta, es $\inf \sigma(T) = \inf_{\|x\| = 1} (Tx,x)$ y por lo tanto tenemos que $T \geq 0$ si y sólo si $\sigma(T) \subset [0,+\infty)$. 
Esto nos permite hacer la misma construcción que antes para este caso, y como ahora es $f_n \geq 0$, de aquí se concluye que $T^{1/n} \geq 0$. Para la unicidad resta notar que si $\widetilde{S} \geq 0$ es otra raíz $n$-ésima, entonces $\widetilde{S} = |\widetilde{S}| = (\widetilde{S}^n)^{1/n} = T^{1/n}$.
\end{itemize}

Finalmente, como para todo $n \geq 1$ es $f_n(0) = 0$, sabemos que $T^{1/n}$ siempre resulta compacto.
\end{proof}

La siguiente aplicación es una adaptación del $\tpaint{Teorema 12.44}$ de $\cite{Rudin}$, que a su vez es una versión del teorema ergódico medio de Von Neumann para transformaciones unitarias:

\begin{tcolorbox}
\begin{theorem*} Si $H$ es un espacio de Hilbert y $U \in \mathscr{L}(H)$ una transformación unitaria, entonces para cada $x \in H$ los \textit{promedios} $\frac{1}{n}(x + Ux + \dots + U^{n-1}x)$ convergen puntualmente a un elemento $y \in H$.
\end{theorem*}
\end{tcolorbox}
La demostración presente en $\cite{Rudin}$ hace uso del teorema espectral en un caso más general. Usando que los autovalores de una transformación unitaria yacen en el círculo de radio $1$, el teorema se reduce a un cálculo directo de convergencia puntual para una cierta sucesión de funciones.

Siguiendo la idea de esta demostración pero en el caso de operadores compactos y autoadjuntos, definimos a continuación el concepto de medida espectral y con esto probamos un resultado auxiliar de convergencia. 

Concluimos con el $\tpaint{Teorema 2.0.4}$, el cual afirma que si $T$ es un operador compacto, autoadjunto y contractivo, entonces sus \textit{promedios} convergen puntualmente a la proyección ortogonal del subespacio de sus puntos fijos.
\\
\begin{tcolorbox}
\begin{theorem*}[\textbf{Riesz-Markov-Kakutani}] Sea $X$ un espacio topológico Hausdorff y localmente compacto. Si $\psi : C(X) \to \R$ es un funcional lineal positivo, existe una única medida Borel regular $\mu$ en $X$ tal que
\begin{align*}
\psi(f) = \int_X f d\mu.
\end{align*}
para toda $f \in C(X)$.
\end{theorem*}
\end{tcolorbox}

\begin{definition} Sea $H$ un espacio de Hilbert y $T : H \to H$ un operador compacto y autoadjunto. Para cada $h \in H$, la aplicación 
\begin{align*}
f \in C(\sigma(T)) \mapsto \ip{f(T)h,h} \in \R
\end{align*}
resulta un funcional lineal positivo. El teorema de Riesz-Markov-Kakutani nos asegura entonces que existe una única medida Borel regular $\mu_h$ en $\sigma(T)$ que satisface
\begin{align*}
\ip{f(T)h,h} = \int_{\sigma(T)} f d\mu_h
\end{align*}
para toda $f : \sigma(T) \to \R$ continua. Llamamos a $\mu_h$ la \textbf{medida espectral de $T$ asociada a $h$}. 

Esto permite definir $g(T)$ para $g : \sigma(A) \to \R$ una función Borel medible: definimos $Q(x) = \int_{\sigma(T)}gd\mu_x$, y a partir de esto una forma bilineal que luego (fijando $x$ en una coordenada y usando el teorema de representación de Riesz) definirá $g(T)x$ para cada $x$. 

Además, en vista de la proposición que sigue veremos que cuando $g$ es acotada $g(A)$ coincide con la anterior definición, ya que ambas construcciones dan el mismo operador en cada autoespacio de $T$.
\end{definition}
\newpage

\begin{proposition}
Sea $H$ un espacio de Hilbert y $T \in \mathscr{K}(H)$ un operador compacto y autoadjunto. Si $(g_n)_{n \geq 1} \subset \mathcal{C}(\sigma(T),\R)$ es una sucesión uniformemente acotada que converge puntualmente a cierta función $g : \sigma(T) \to \R$, la sucesión de operadores $\{g_n(T)\}_{n \geq 1} \subset \mathscr{L}(H)$ \textbf{sot}-converge a $g(T)$.
\end{proposition}
\begin{proof} Observemos que por el teorema de convergencia dominada, para cada $h \in H$ es
\begin{align*}
(g_n(T)h,h) = \int_{\sigma(T)}g_n d\mu_h \to \int_{\sigma(T)}g d\mu_h = (g(T)h,h).
\end{align*}
Usando la identidad de polarización, vemos que $(g_n(T)x,y) \to (g(T)x,y)$ para todo $x,y \in H$. Por lo tanto tenemos convergencia débil,
\begin{align*}
g_n(T)x \rightharpoonup g(T)x
\end{align*}
para cada $x \in H$. Para terminar alcanza ver que siempre es $\|g_n(T)x\| \to \|g(T)x\|$. Equivalentemente, veremos que $\|g_n(T)x\|^2 \to \|g(T)x\|^2$ para todo $x \in H$.

Por hipótesis sabemos que las funciones $(g_n^2)_{n \geq 1}$ también están uniformemente acotadas y convergen puntualmente a $g^2$, así que el argumento anterior nos dice que efectivamente
\begin{align*}
\|g_n(T)x\|^2 = \ip{g_n(T)x,g_n(T)x} = \ip{g_n(T)g_n(T)x,x} = \ip{g_n^2(T)x,x} \to \ip{g^2(T)x,x} = \|g(T)x\|^2
\end{align*}
para cada $x \in H$.
\end{proof}

\begin{theorem}[un caso particular del teorema ergódico medio de Von Neumann] Sea $H$ un espacio de Hilbert. Si $T \in \mathscr{K}(H)$ un operador compacto y autoadjunto tal que $\|T\| \leq 1$, entonces los promedios de $T$ \textbf{sot}-convergen al proyector $\pi_T$ del subsepacio de puntos fijos de $T$. Es decir, si notamos $E_1 = \{x \in H : Tx = x\}$ y $\pi_T := P_{E_1}$, entonces
\begin{align*}
\frac{1}{n}\sum_{i=1}^{n}T^ix \ \xrightarrow{n \to \infty} \ \pi_Tx.
\end{align*}
para todo $x \in H$.
\end{theorem}
\begin{proof} Notemos en primer lugar que $\sigma(T) \subset [-\|T\|,\|T\|] \subset [-1,1]$. Para cada $n \in \N$, definimos $g_n(x) := \frac{1}{n}\sum_{i=1}^nx^i$ para cada $x \in [-1,1]$. Tenemos así que $\frac{1}{n}\sum_{i=1}^{n}T^i = g_n(T)$. Por otro lado, la proyección $\pi_T$ coincide con la evaluación en $T$ de 
\begin{align*}
g(x) := \begin{cases}
1 &\text{si $x = 1$}\\
0 &\text{en caso contrario}
\end{cases}
\end{align*}

En vista de la $\tpaint{Proposición 2.0.4}$, basta probar que la sucesión $(g_n)_{n \geq 1}$ está uniformemente acotada y converge puntualmente a $g$. Lo primero se deduce de que si $x \in [-1,1]$ entonces
\begin{align*}
|g_n(x)| \leq \frac{1}{n}\sum_{i=1}^n|x|^i \leq \frac{1}{n}\sum_{i=1}^n1 = 1.
\end{align*}

Ahora veamos la convergencia puntual. En primer lugar, la sucesión $(g_n(1))_{n \geq 1}$ es constantemente $1$ y por lo tanto converge a $g(1) = 1$. Por otro lado, sabemos que $g_n(-1)$ es cero para $n$ par y $-1/n$ para $n$ impar. De aquí se ve que entonces que $g_n(-1) \to 0 = g(-1)$. Finalmente, si $\lambda \in (-1,1)$ entonces
\begin{align*}
|g_n(\lambda)| \leq \frac{1}{n}\sum_{i=1}^n|\lambda|^i \leq \frac{1}{n}\sum_{i \geq 0}|\lambda|^i = \frac{1}{n} \cdot \frac{1}{1-|\lambda|} \to 0.
\end{align*}
Consecuentemente, debe ser $g_n(\lambda) \to 0 = g(\lambda)$.
\end{proof}

\begin{thebibliography}{}
\bibitem{Brezis} H. Brezis. \textit{Functional Analysis, Sobolev Spaces and Partial Differential Equations}. Springer, 2010.
\bibitem{ReedSimon} M. Reed, B. Simon. \textit{Methods of Modern Mathematical Physics}.  Academic Press Inc., 1980.
\bibitem{Rudin} W. Rudin. \textit{Functional Analysis}. International Series in Pure and Applied Mathematics, McGraw-Hill, 1991.
\bibitem{Teschl} G. Teschl. \textit{Topics in Real and Functional Analysis}, versión del 9/7/19 (https://www.mat.univie.ac.at/~gerald/ftp/book-fa/fa.pdf).

\end{thebibliography}
\end{document}


